\documentclass{article}
% Change "article" to "report" to get rid of page number on title page
% \usepackage{amsmath,amsfonts,amsthm,amssymb}
\usepackage{setspace}
\usepackage{fancyhdr}
\usepackage{lastpage}
\usepackage{extramarks}
\usepackage{textcomp}
\usepackage{amsmath}
\usepackage{lstcustom}
\usepackage{enumitem}
\usepackage{multicol}
\usepackage{url}
\usepackage{tikz}

\usepackage{totcount} % For the total points


% In case you need to adjust margins:
\topmargin=-0.45in
\evensidemargin=0in
\oddsidemargin=0in
\textwidth=6.5in
\textheight=9.0in
\headsep=0.25in

% Commands for points
\newtotcounter{points}
\newcommand*{\totalpoints}{\thepoints}
\newcommand*{\pts}[1]{\addtocounter{points}{#1}(#1pt)}

\newtotcounter{ecpoints}
\newcommand*{\totalecpoints}{\theecpoints}
\newcommand{\ecpts}[1]{\addtocounter{ecpoints}{#1}(#1pt)}

% Exercise Specific Information
\newcommand{\hmwkTitle}{\underline{Exam}}
\newcommand{\hmwkDueDate}{2016-07-09 8:00 a.m.}
\newcommand{\hmwkClass}{COS 101} \newcommand{\hmwkPoints}{
  \ref{lastQuestion} questions; \protect\total{points} pts +
  \protect{\total{ecpoints}} ec; \pageref{LastPage} pgs.}

\newcommand{\cstart}{\vspace{.4cm}}

% Setup the header and footer
\pagestyle{fancy}
\lhead{\hmwkClass\ \hmwkTitle} 
\chead{\hmwkPoints}
\rhead{\hmwkDueDate}
\lfoot{\lastxmark}  
\cfoot{}            
\rfoot{Page\ \thepage\ of\ \pageref{LastPage}}
\renewcommand\headrulewidth{0.4pt} 
\renewcommand\footrulewidth{0.4pt}

% Setup listing look

\lstset{
  language=C++,
  style=eclipse,
  showspaces=false, 
  numbers=left,
  frame=tb,
}

\begin{document}

\begin{spacing}{1.4}

\begin{enumerate}[leftmargin=*]
\item \pts{1} \textbf{Name:} \hrulefill

Carefully read each question, and write the answer in the space
provided.  If answers are written obscurely, zero credit will be
awarded. Use the notations of class and the book. You may use your
book and notes during the test, but electronic devices are
\textbf{not} allowed. Questions may be brought to the
instructor. Print your name in the space provided above.

For all questions, if the answer to a question should be a memory
location, use the \textbf{address operator} followed by the
variable. If the address location is unknown, write the
\textbf{address operator} followed by the word \textbf{UNKNOWN}.
  

\item What is the value of each expression using the provided
  variables. Place a decimal point in your answer to indicate a double
  value (eg. 2.0).

\begin{lstlisting}
double x = 100.0;
double y = 15;
int m = 25;
int n = 10;
\end{lstlisting}

\begin{enumerate}

\item \pts{1} \underline{\hspace{1in}} \lstinline$x / n + m$

\item \pts{1} \underline{\hspace{1in}} \lstinline$n / x + m$

\item \pts{1} \underline{\hspace{1in}} \lstinline$int(x) / n + m$

\item \pts{1} \underline{\hspace{1in}} \lstinline$n / (int)x + m$

\item \pts{1} \underline{\hspace{1in}} \lstinline$int(y) % m$

\end{enumerate}

\item \pts{5} Write a \textbf{templated} function that converts an
  array into a string.  For example, if the array contained the
  numbers $0,3,4,6$, the expected output would be \lstinline$"[ 0, 3, 4, 6 ]"$. 
  You may assume that the type in the array has a
  stringstream insertion-operator function defined. (Hint: hw03)

\newpage

\hspace{-.5cm}\textbf{Questions \ref{sl1}-\ref{el1}} refer to the code shown below. 

\begin{lstlisting}[showspaces=false,showlines=true,escapeinside={*@}{@*}]
int i, *i1;
double* d = new double[5];
double* d2 = new double[10];
i = 0;
i1 = &i;
while (i < 5) {
  d[i] = i;
  *(d2 + (i * 2)) = i;
  i = i + 2;
}
*@\label{firstl1}@*
delete [] d;

\end{lstlisting}


\item\label{sl1} \pts{1} How many variables \textbf{are} declared as
pointers? \underline{\hspace{1in}}

\item \pts{1} How many variables are \textbf{not} declared as
pointers? \underline{\hspace{1in}}

\item \pts{2} How many variables have values that are memory locations
on the heap at \textbf{line \ref{firstl1}}? \underline{\hspace{1in}}

\item \pts{1} Is dynamic memory freed? If so, indicate which lines it
occurs. \underline{\hspace{1in}}

\item \pts{2} Does the example contain a memory leak? (yes/no)
\underline{\hspace{1in}}

\item \pts{2} If the example contains a memory leak, show in the space
  provided below how it can be corrected without changing the
  resulting variable values. If it does not contain a memory leak,
  explain why there is none.

\vspace{2cm}

\item \pts{2} What is the value of \lstinline$*d$ at \textbf{line
  \ref{firstl1}}? \underline{\hspace{1in}}

\item \pts{2} What is the value of \lstinline$*i1$ at \textbf{line
  \ref{firstl1}}? \underline{\hspace{1in}}

\item \pts{2} What is the value of \lstinline$i1$ at \textbf{line
  \ref{firstl1}}? \underline{\hspace{1in}}

\item \pts{2} What is the value of \lstinline$d[2]$ at \textbf{line
  \ref{firstl1}}? \underline{\hspace{1in}}

\item\label{el1} \pts{2} What is the value of \lstinline$d2[2]$ at \textbf{line
  \ref{firstl1}}? \underline{\hspace{1in}}

\newpage

\hspace{-.5cm}\textbf{Questions \ref{rl1}-\ref{rel1}} refer to the code shown below. 

\begin{lstlisting}[showspaces=false,showlines=true]
int f(int n) {
  if (n == 0)
    return 1;
  else
    return n * f(n-1);
}
\end{lstlisting}

\item\label{rl1} \pts{2} Show the result of evaluating the function
  \textbf{f} for the values: 0, 2, 3, and 5.

\vspace{4cm}

\item \pts{1} The line number of the return statement for the base
  case in the function \textbf{f} is \underline{\hspace{1in}}.

\item\label{rel1} \pts{1} The line number for the recursive case in
  the function is \underline{\hspace{1in}}.


\hspace{-.5cm}Use the following listing to answer the question below.

\begin{multicols}{2}
\begin{lstlisting}[numbers=none,frame=none]
class Node {
public:
  int x;
  Node* next;
};
class Stack {
  Node* top;
public:
  void push(int val);
};
\end{lstlisting}
\end{multicols}


\item \pts{4} Convert the \lstinline$Node$ and \lstinline$Stack$ class
  specifications, as shown above, into templated classes.


\vspace{7cm}

\newpage

\item\label{ab} \pts{10} Assuming that the classes A and B have been defined as
  shown below, what does the following program display as output?
  (Write output on lines below.) 

\begin{multicols}{2}
\begin{lstlisting}[numbers=none,frame=none]
class A {
  public:
  virtual string m() const {
    return "A";
  }
};

class B: public A {
  public:
  virtual string m() const {
    return "B";
  }
};
\end{lstlisting}
\end{multicols}
\begin{lstlisting}[escapeinside={*@}{@*}]
void f1(A a) { cout << a.m() << endl; }

void f2(A& a) { cout << a.m() << endl; } 

int main() {
  A* a = new A();
  B* b = new B();

  cout << a->m() << endl;*@\label{l1}@*
  cout << b->m() << endl;*@\label{l3}@*
  a = b;
  cout << a->m() << endl;*@\label{l5}@*
  
  f1(*a);*@\label{l7}@*
  f2(*a);*@\label{l8}@*

  return 0; 
}
\end{lstlisting}

\newcommand{\lin}[1]{
\par\smallskip\noindent\parbox[t]{.09\textwidth}{\raggedright\textbf{Line #1:}}
 \parbox[t]{.3\textwidth}{\raggedleft\hrulefill}\par\smallskip\vspace{1em}
}%

\begin{multicols}{2}

\lin{\ref{l1}}

\lin{\ref{l3}}

\lin{\ref{l5}}

\lin{\ref{l7}}

\lin{\ref{l8}}

\end{multicols}

\newpage

\begin{lstlisting}
A* a = new B();
B* b = dynamic_cast<B*>(a);

if (b == NULL)
  cout << "I'm NOT a B object?" << endl;
else
  cout << "I'm a B object!" << endl;
\end{lstlisting}

\item \pts{2} Assuming the definitions from question \ref{ab}, write
  the output from the code shown above.

\vspace{2cm}

\item \pts{4} Depict the binary search tree (BST) generated by
inserting the following keys: 3, 1, 6, 4, 10, 2, 5, 9, 8, 7. You need
only show the final tree.

\vspace{6cm}

\item\label{fbst} \pts{3} Depict the BST after removing 8, 6, and
1. Depict the BST after each removal is completed. Use the removal
method used in class and in the book.

\vspace{12cm}

\item \pts{2} List the \textbf{preorder} traversal of the final tree
in question \ref{fbst}.

\vspace{2cm}

\item \pts{2} List the \textbf{postorder} traversal of the final tree
in question \ref{fbst}.

\vspace{2cm}

\item \pts{2} List the \textbf{inorder} traversal of the final tree in
question \ref{fbst}.

\vspace{2cm}

\item \pts{2} List the \textbf{breadth-first} traversal of the tree in
question \ref{fbst}.

\vspace{2cm}

\item\label{tree} \pts{3} For each node in the following tree,
  write the height of the subtree rooted at its node to the left and
  write the AVL balance factor to the right.

\begin{center}

\tikzset{
  tnode/.style = {text centered, font=\sffamily, circle, black, draw=black, inner sep=2pt, minimum width=2em}
}


\resizebox{12.0cm}{!}{%
\begin{tikzpicture}[level/.style={sibling distance = 16cm/(2^#1),
  level distance = 2cm}] 

\node [tnode] { 18 }
  child { node [tnode] { 10 } 
    child { node [tnode] { 5 } 
      child { node [tnode] { 2 } }
      child[missing]{}
    }
    child { node [tnode] { 15 } 
      child { node [tnode] { 13 } }
      child[missing]{}
    }
  } 
  child { node [tnode] { 23 }
    child { node [tnode] { 20 } }
    child[missing]{} 
  }
;
\end{tikzpicture}
}%
\end{center}

\newpage

\item \pts{2} Is the tree of question \ref{tree} a \textbf{binary heap}?

\vspace{1cm}

\item \pts{2} Is the tree of question \ref{tree} a \textbf{binary search tree}?

\vspace{1cm}

\item \pts{2} Is the tree of question \ref{tree} an \textbf{AVL tree}?

\vspace{1cm}


\item \pts{8} Depict the \textbf{AVL tree} generated by inserting the
  following keys: 3, 1, 6, 4, 10, 2, 5, 9, 8, 7. Clearly label and
  depict each rotation with the proper rotation from the AVL cheat
  sheet (LL,RR,etc.).  You will lose significant points if rotations
  are not clearly marked.

\newpage

\item \pts{5} Show the action of removing 10, 7, and 9
  from the final AVL tree of the previous question.  Depict the AVL
  tree after each removal is completed and label any rotations. Use
  the removal method used in class and in the book.

\vspace{6cm}


\item\label{splaytree} \pts{8} Depict the \textbf{splay tree} generated by
  inserting the following keys into an initially empty tree: 10, 25,
  1, 14, 17, 86, 15, 2, 6, 13. Clearly label any splay operations that
  occur.

\newpage

\begin{center}
\textit{Additional space for the previous question}
\end{center}

\vspace{8cm}

\item\label{lastQuestion} \pts{5} Show the action of performing
  searches on the final splay tree of the previous question for keys
  17, 14, and 2.

\newpage

\textbf{Extra Credit}: \ecpts{5} Depict all non-isomorphic \textbf{binary
search trees} containing keys from the set ${1,2,3,4}$.  If the number
of trees is too large to depict, explain how to calculate the number
of trees.

\vspace{12cm}

\textbf{Extra Credit}: \ecpts{5} Depict all non-isomorphic \textbf{AVL trees}
containing keys from the set ${1,2,3,4}$.  If the number of trees is
too large to depict, explain how to calculate the number of trees.


\end{enumerate}
\end{spacing}

\end{document}


